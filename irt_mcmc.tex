\documentclass{article}
\usepackage{amsmath}
\usepackage{amssymb}

\newcommand{\Normal}{\mathcal{N}}
\newcommand{\G}{\mathcal{G}}
\newcommand{\R}{\mathbb{R}}
\newcommand{\bX}{\mathbf{X}}
\newcommand{\bY}{\mathbf{Y}}
\newcommand{\bt}{\boldsymbol\tau}
\newcommand{\st}{\sigma_{\ta}^2}
\newcommand{\ta}{\theta}

\begin{document}

\title{Item Response Theory - MCMC Formulation}
\author{Oren Livne}
\date{\today}

\maketitle

\begin{abstract}
  We describe a basic Item response theory model for the National Assessment of
  Educational Progress (NAEP) data, and the standard Monte Carlo Markov Chain (MCMC)
  algorithm for estimating of target quantities, which in this case are
  performance metrics of different student populations.
\end{abstract}

\section{Model}
Item response theory (IRT) (also known as latent trait theory) is a psychometric paradigm for the design, analysis, and scoring of tests, questionnaires, and similar instruments measuring abilities, attitudes, or other variables. It is a theory of testing based on the relationship between individuals' performances on a test item and the test takers' levels of performance on an overall measure of the ability that item was designed to measure.

The NAEP model contains several assumptions (e.g., specific prior distributions on student
latest abilities and item asymptotes \cite{matt02}). The model described here is a simplified
version that more or less follows \cite[Sec. 1.7]{junker}. This version should (a) be a stepping
stone towards more comprehensive models of NAEP data that still contains the main computational
challenge of speedinfg up MCMC convergence; and (b) provide more accurate and fair estimates
of the parameters and target quantities. Instead of placing assumptions on the distribution of
latest ability as a function of background variables, we propose to estimate latest abilities
without presumptions in the MCMC, then calculate their statistics and correlation with
the background variables. 

\subsection{Observed Data}
A test is a set of $M \approx 200$ items. For simplicity, we assume all items are multiple-choice. $N \approx 10^5$ persons take the test. Each person $p$ answers a different subset $I_p$ of size $|I_p| \approx 40$ items (some persons may share the same subset). Person $p$'s response to item $j$ is denoted by the binary $X_{pj}$ ($X_{pj}=1$ is the answer was correct, $0$ otherwise). 

Also available is a matrix $\bY \in \R^{N \times B}$ of background variables for all persons.

\subsection{Unknowns}
The test measures a latent ability $\ta \in \R^K$. In the ETS NAEP data, $K=5$. $\ta_p$ denotes the unknown ability of student $p$

A person's response to item $i$ is modeled by an Item Response Function (IRF). A basic choice is a {\it three-parameter logistic model (3PL)}. The probability of a correct response is assumed to be
\begin{equation}
  P_i(\ta_p; \beta_i) = P[X_{pi} = 1 | \ta_p,\beta_i] = \pi_i + (1-\pi_i) \frac{e^{-a_i (\ta - b_i)}}{1 + e^{-a_i (\ta - b_j)}}\,.
\end{equation}
We fix $\pi_j \equiv 0.2$ (assuming five-choice items). The unknown item parameters $\beta_i := (a_j, b_j)$ represent the discrimination and difficulty of the item, respectively. The density for $X_{pi}$ is
\begin{equation}
  f(x_{pi}|\ta_p;\beta_i) = P_i(\ta_p;a_i,b_i)^{x_{pi}} \left(1 - P_i(\ta_p;a_i,b_i)\right)^{1-x_{pi}}\,.
\end{equation}
We denote by $\bt := (\ta_1, \dots, \ta_N, \beta_1, \dots, \beta_M)$ the concatenated vector of all unknowns.

\subsection{Posterior Distribution}
By Bayes' theorem, the posterior distribution of parameters given the responses is proportional to
\begin{equation}
  f(\bX|\bt) f(\bt) = \prod_{p=1}^N \left\{ \prod_{i \in I_p} f(X_{pi}|\ta_p,\beta_i) \right\} f_p(\ta_p|\st) f(\st)\,.
  \label{model}
\end{equation}
The $p$ is due to the experimental independence assumption in IRT and the product over $i$ is due to IRT’s local independence assumption. It is also typical to specify independent priors for parameters. In (\ref{model}) we assumed an informative prior only on $\ta_p$ with the hyperparameter $\st$ whose hyper-distribution is $f(\st)$. For convenience, we add $\st$ as the last entry of $\bt$, which has a total of $T := K P + 2 I + 1$ elements. The following distributions are used, and fully specify the model:
\begin{eqnarray}
	X_{pi} &\sim& {\mbox{Bernoulli}}(y_{pi})
	\label{dist_x} \\
	\ln \frac{y_{pi}}{y_{pi}+1} &\sim& a_i (\ta_p - b_i)
	\label{dist_y} \\
	\ta_p &\sim& \Normal(0, \st)
	\label{dist_ta} \\
	\st &\sim& IG(\alpha_{\ta}, \beta_{\ta})
	\label{dist_st} \\
	\alpha_{\ta}, \beta_{\ta} && {\mbox{pre-specified: }} \alpha_{\ta} = \beta_{\ta} = 1
	\label{dist_ab}
\end{eqnarray}
for all $p=1,\dots,P$ and $i=1,\dots,I$. $IG$ denotes the inverse gamma distribution.

\subsection{Target Quantities}
One of NAEP's goals is to measure the mean of some function $g(\cdot)$ of proficiency scores $\ta$ over all individuals in some sub-population $\G$. The Nation's Report Card reports results on the composite scale, which is a weighted average of sub-scales, $g(\ta) = a^T \ta$ for some fixed $a$. The Nation's Report Card also reports the proportion of students in $\G$ whose composite score is in some pre-defined achievement range, i.e.,
\begin{equation}
	g(\ta) = I_{(l,u)}(a^T \ta) =
  \begin{cases}
    1, & a^T \ta \in (l,u), \\
    0, & a^T \ta \not \in (l,u)\,.
  \end{cases}
\end{equation}

\section{MCMC Algorithm}
Thanks to the factorized form of (\ref{model}), it follows from (\ref{model})--(\ref{dist_ab}) that the complete conditional densities for the individual parameters are
\begin{eqnarray}
  f(\ta_p|rest) &\propto& \prod_{i \in I_p} 
  P_i(\ta_p;a_i,b_i)^{x_{pi}} 
  \left(1 - P_i(\ta_p;a_i,b_i)\right)^{1-x_{pi}} 
  \Normal(\ta_p|0,\st)\,, \forall p\,,
  \label{cond_ta}
  \\
  f(a_i|rest) &\propto& \prod_{p} 
  P_i(\ta_p;a_i,b_i)^{x_{pi}} 
  \left(1 - P_i(\ta_p;a_i,b_i)\right)^{1-x_{pi}} 
  \Normal(\ta_p|0,\st)\,, \forall I\,,
  \label{cond_a}
  \\
  f(b_i|rest) &\propto& \prod_{p} 
  P_i(\ta_p;a_i,b_i)^{x_{pi}} 
  \left(1 - P_i(\ta_p;a_i,b_i)\right)^{1-x_{pi}} 
  \Normal(\ta_p|0,\st)\,, \forall I\,,
  \label{cond_b}
  \\
  f(\st|rest) &\propto& \prod_{p} 
  \Normal(\ta_p|0,\st) IG(\st|\alpha_{\ta},\beta_{\ta}) \\
  &=& IG\left(\st|\alpha_{\ta} + \frac{P}{2}, \beta_{\ta} + \frac12 \sum_p \ta_p^2\right)\,.
  \label{cond_st}
\end{eqnarray}
A Hastings-Metropolis step for a parameter $\tau_t$, $t=1,\dots,T$ requires a proposal distribution $q(\tau^*|tau_t)$. Assuming symmetric $q$, the step is
\begin{enumerate}
	\item Generate a candidate $\tau^*$ using the probability density $q(\tau^*|tau_t)$.
	\item Calculate the acceptance probability
	\begin{equation}
		\alpha_* := \min\left\{1, \frac{f(\tau^*|rest)}{f(\tau_t|rest)} \right\}\,.
	\end{equation}
	\item Set $\tau_t \leftarrow tau^*$ with probability $\alpha^*$, otherwise leave $\tau_t$ unchanged.
\end{enumerate}

\section{Fast MCMC - Questions/Ideas}
\begin{itemize}
	\item How to measure convergence, i.e., equilibration = convergence to the stationary distribution? 
	\item How to efficiently generate independent configurations from the stationary distribution once we converged
	(so that we can estimate target quantities with a reasonably-sized sample)?
	\item Initial guess for parameters: in general, by continuation. But one should define the coarse variables
	to be able to continue from a coarser scale, and we don't have them yet. So, one could define the $\ta$'s
	from the fraction of correct responses of a student on the corresponding questions (relative to the general
	population average, which is defined as $\ta=0$ by our model).
	\item Find a coarse variable set using clustering. $\ta_p$'s should probably be clustered separately. 	Possibly also $a_i$'s and $b_i$'s; or one could just coarsen all of the $\beta_i$'s.
	\item Measure coarse variable set quality by Compatible MC (CMC)'s convergence rate. Since this should be a fast convergence for a good set, convergence of target quantities may suffice to define the convergence rate here.
\end{itemize}

\section{Generalizations}
\begin{itemize}
	\item Let the item guessing probability $\pi_j$ be a model parameter instead of fixed. In the NAEP model,
	the prior is $\pi_j \sim {\mbox{Beta}}(a_{\pi}, b_{\pi})$ where $a_{\pi}=10, b_{\pi}=40$ might be used for a five
	choice item \cite{matt02}. Apparently, varying $\pi$ gives rise to a different correlation structure among the $a$'s and $b$'s than for fixed $\pi$.
	\item Add constructed response item type.
\end{itemize}

\bibliographystyle{alpha}
\bibliography{irt_mcmc}

\end{document}